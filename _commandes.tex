%%%% couleurs
\definecolor{darkgreen}{rgb}{0.0, 0.40, 0.20}
\definecolor{darkcyan}{rgb}{0.0, 0.55, 0.50}


%%%% pagination et sections
\newcommand{\titre}[1]{
  \begin{center}
    \textsf{\bfseries \Large #1}
  \end{center}
}
\newcommand{\sousTitre}[1]{
  \textsf{\bfseries #1}
}
\newcommand{\pasDePagination}{
  \thispagestyle{empty}
}


%%%% activité ou TP
\newcounter{acti}
\newcommand{\titreActi}[2]{
  \refstepcounter{acti}
  \titre{#1 \arabic{acti} : #2}
}
\newcommand{\titreTP}[1]{
  \titreActi{TP}{#1}
}
\newcommand{\titreActivite}[1]{
  \titreActi{Activité}{#1}
}
\newcommand{\titreEvaluation}[1]{
  \titreActi{Évaluation}{#1}
}


%%%% chapitre, section et sous-section
\newcommand{\titreChapitre}[1]{
  \titre{Chapitre \arabic{section} : #1}
}
\newcommand{\titrePartie}[1]{
  \vspace*{24pt}
  \refstepcounter{subsection}
  \sousTitre{\Large \Roman{subsection} -- #1}
  \vspace*{10pt}
}
\newcounter{sousSection}
\newcommand{\titreSection}[1]{
  \vspace*{16pt}
  \refstepcounter{subsubsection}
  \setcounter{sousSection}{0}
  \sousTitre{\large \arabic{subsubsection} -- #1}
  \vspace*{10pt}
}
\newcommand{\titreSousSection}[1]{
  \vspace*{12pt}
  \refstepcounter{sousSection}
  \sousTitre{\arabic{subsubsection}.\arabic{sousSection} -- #1}
  \vspace*{8pt}
}
%%%% fixe le numéro de l'activité
\newcommand{\numeroActivite}[1]{
  \setcounter{exercice}{0}
  \setcounter{subsection}{0}
  \setcounter{figure}{0}
  \setcounter{acti}{#1 - 1}
}


%%%% lignes
\newcommand{\ligne}{
  \par\noindent\rule{\textwidth}{0.4pt}
}
\newcommand{\lignePointillee}[1]{
  \makebox[#1\textwidth]{\dotfill}
}


%%%% en-tête
\newcommand{\teteGauche}[2]{
  \lhead{
    \textbf{\footnotesize #1}
    \newline
    \footnotesize #2
  }
}
\newcommand{\teteDroite}[2]{
  \rhead{
    \hfill \textbf{\footnotesize #1}
    \newline \hfill
    \footnotesize #2
  }
}
\newcommand{\enTete}[2]{
  \pagestyle{fancy}
  \setcounter{section}{#2}
  \teteGauche{Lycée Condorcet}{Chapitre \arabic{section} -- #1} % left header
  \chead{} % central header
  \teteDroite{2021 -- 2022}{Seconde}
}
%
\newcommand{\enTeteFicheReussite}{
  \newpage
  \setcounter{subsection}{0}
  \pasDePagination
  
  \phantom{b}
  \vspace*{-60pt}
  \titre{Fiche \og Réussir son évaluation \fg}
  
  \titrePartie{Ce que je dois savoir}
  
  Pour savoir quoi réviser, je lis les points clés du chapitre évalués :
  \begin{itemize}
    \item Si je pense maîtriser une notion, je coche la case \ok
    \item Si je pense que je dois la retravailler, je coche la case \pasOk
  \end{itemize}
  
  Pour travailler les notions qui ne sont pas maîtrisées, je reprend les activités associés.
}
\newcommand{\basDePageFicheReussite}{
  \titrePartie{Ce qu'il me reste à faire}
  
  Pour être sûr-e d'obtenir une bonne note, je m'entraîne avec les exercices corrigés du manuel indiqués dans la colonne de droite.
}


%%%% exercice
\newcounter{exercice}
\newcommand{\exo}[2]{
  \refstepcounter{exercice}
  \hspace{24pt}
  \textbf{\arabic{exercice} --}
  #1
  
  \vspace*{8pt}
  \reponse{#2}
  \ifnum #2 > 0
    \vspace*{-12pt}
  \fi
}
% reponse
\newcounter{int}
\newcommand{\reponse}[1]{
  \setcounter{int}{-1}
  \loop
  \stepcounter{int}
  \ifnum \value{int} < #1
  \lignePointillee{0.99} \\[8pt]
  \repeat
}


%%%% simple boite
\newenvironment{boite}{
  \begin{tcolorbox}
  [ breakable, enhanced jigsaw, % to break box over page
    arc = 0mm, % straight line
    colback= white, % white background
    colframe= black % dark frame
  ]
}
{ \end{tcolorbox} }


%%%% documents
\newcounter{document}
\newcommand{\titreDocu}[1]{
  \refstepcounter{document} % update counter
  \textbf{Document \arabic{document} -- #1} % print document title
  \addcontentsline{toc}{document}{\protect\numberline{} #1} % update table of content
}
\newenvironment{doc}[1]{
  \begin{boite}
    \titreDocu {#1} \newline
}
{ \end{boite} }


%%%% objectifs
\newenvironment{objBoite}{
  \begin{tcolorbox}
  [boxrule = 0pt,
  frame hidden, sharp corners,
  colback = white, enhanced,
  borderline north={2pt}{0pt}{darkcyan},
  borderline south={2pt}{0pt}{darkcyan},
  borderline west={4pt}{0pt}{darkcyan},
  borderline east={4pt}{0pt}{darkcyan}
  ]
  
}
{
  \end{tcolorbox}
}
\newenvironment{objectifs}{
  \begin{objBoite}
    \begin{center}
      \sousTitre{\Large Objectifs de la séance :}
      \begin{listeObjectifs}
}
{ 
      \end{listeObjectifs}
    \end{center}
  \end{objBoite}
}


%%%% Tableau de competence
\newenvironment{tableauCompetences}{
  \centering
  \setlength{\extrarowheight}{6pt}
  \tabularx{\linewidth}{| c | X | c | c | c | c |}
    \hline
    \rowcolor{gray!20}
    \centering \textbf{Compétences} & \centering \textbf{Items} & \textbf{D} & \textbf{C} & \textbf{B} & \textbf{A}
    \\ \hline
}
{
    \\ \hline
  \endtabularx
}

%%%% Tableau de connaissances
\newenvironment{tableauConnaissances}{
  \centering
  \setlength{\extrarowheight}{6pt}
  \tabularx{\linewidth}{| X | c | c | c | c |}
    \hline
    \rowcolor{gray!20} 
    \textbf{Points clés du chapitre} & \ok & \pasOk & \textbf{En classe} & \textbf{Exercices}
    \\ \hline
}
{   
    \\ \hline
  \endtabularx
}

%%%% Espace pour une appréciation
\newcommand{\appreciation}{
  \vspace*{8pt}
  \sousTitre{\large Appréciation}
  \vspace*{-4pt}
  \begin{boite}
    \vspace{100pt}
    \phantom{b}
  \end{boite}
}


%%%% symboles : chevron, flèche, attention, etc.
\newcommand{\chevron}{
  \textcolor{darkcyan}{\faChevronRight}
}
%
\newcommand{\fleche}{
  \textcolor{darkcyan}{\faCaretRight}
}
%
\newcommand{\attention}{
  \textcolor{darkcyan}{\faExclamationTriangle}
}
%
\newcommand{\flecheLongue}{
  \textcolor{darkcyan}{\faLongArrowRight}
}
%
\newcommand{\ok}{
  \textcolor{darkcyan}{\faCheckCircle}
}
%
\newcommand{\pasOk}{
  \textcolor{darkcyan}{\faTimesCircle}
}


%%%% problematique
\newcommand{\problematique}[1]{
  {
    \large
    \fleche
    \emphase{Problématique :} \newline
    \textbf{#1}
  }
}


%%%% Passage important
\newenvironment{encart}{
  \begin{tcolorbox}
  [boxrule = 0pt,
  frame hidden, sharp corners,
  colback = white,
  enhanced, borderline west={4pt}{0pt}{darkcyan}]
  
}
{
  \end{tcolorbox}
}


%%%% emphase
\newcommand{\emphase}[1]{
  \textcolor{darkcyan}{\textsf{\bfseries \large #1}}
}
%
\newcommand{\important}[1]{
  \!\textcolor{darkgreen}{\textsf{\bfseries #1}}\!
}
%
\newcommand{\exemple}[1]{
  \flecheLongue \textit{#1}
}
%
\newcommand{\note}{
  \textbf{Note :}
}
%
\newcounter{appelProf}
\newcommand{\appelProf}{
  \refstepcounter{appelProf}
  \hspace{24pt} \faHandPaperO \hspace{2pt}
  \textbf{Appel n$^\circ$ \arabic{appelProf} :}
}


%%%% qcm
\newlist{qcm}{itemize}{2}
\setlist[qcm]{label=$\square$, leftmargin=2cm}


%%%% liste d'objectif
\newlist{listeObjectifs}{itemize}{2}
\setlist[listeObjectifs]{label=\chevron}


%%%% unité, degré
\newcommand{\unit}[1]{
  \; \mathrm{#1}
}
\newcommand{\degree}{
  ^\circ\!
}
\newcommand{\Frac}[2]{
  \displaystyle \frac{#1}{#2}
}
\newcommand{\Tfus}{
  T_{\text{f}}
}
\newcommand{\Teb}{
  T_{\text{éb}}
}


%%%% Séparation de la page
\newcommand{\separationTroisBlocs}[3]{
  \begin{minipage}[t]{0.3\linewidth}
    #1
  \end{minipage}
  ~
  \begin{minipage}[t]{0.3\linewidth}
    #2
  \end{minipage}
  ~
  \begin{minipage}[t]{0.3\linewidth}
    #3
  \end{minipage}
}
%%%%
\newcommand{\separationDeuxBlocs}[2]{
  \begin{minipage}[t]{0.48\linewidth}
    #1
  \end{minipage}
  ~
  \begin{minipage}[t]{0.48\linewidth}
    #2
  \end{minipage}
}


%%%% Espace pour indiquer nom, prénom et classe
\newcommand{\nomPrenomClasse}{
  \vspace*{-24pt}
  Nom : \lignePointillee{0.3}
  Prénom : \lignePointillee{0.3}
  Classe : \dotfill
}


%%%%
\newcommand{\competence}[1]{
  {\footnotesize
    \textit{(#1)}
  }
}